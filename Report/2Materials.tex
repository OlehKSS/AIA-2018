\section{Materials}
\label{sec:materials}
\subsection{Database}
This project uses a mammographic research database called INbreast \cite{Moreira2012}. The use of a publicly known database adds legitimacy to our system, since the results can easily be compared to current and future work with this INbreast.\par
The database was acquired between 2008 and 2010, using MammNovation Siemens FFDM. It consists on images from screening, diagnostic, and follow-up exams. In total, the database has 410 images. It also contains samples with various breast densities, one of the main characteristics that affects abnormality detection (breasts with high density are harder to evaluate). In 56 cases, a biopsy was performed, of which 11 were benign and 45 were malignant.\par
The dataset contains examples of normal mammograms, mammograms with masses, mammograms with calcifications, architectural distortions, asymmetries and images with multiple findings. A mass, the subject of this project, is defined by the BI-RADS as a three-dimensional structure demonstrating convex outward borders, usually evident on two orthogonal views \cite{Sickles2013}.\par
The most significant characteristic of this database is the groundtruth annotation. While most of the databases only give a circle around the region of interest (ROI), INbreast provides the exact contour of each mass, made by a specialist in the field and validated by a second specialist. There are 115 masses among 107 images ($\sim$1.1 masses per image). The average mass area is 479 mm$^2$ (standard deviation: 619 mm$^2$, range: 15 mm$^2$ to 3689 mm$^2$). The careful and precise annotation of the contours of the masses is a great tool to simplify performance evaluation. \par
The dataset contains 16-bit images in \texttt{.tif} format, with matrix size ranging from 2560 x 3328 to 3328 x 4084 pixels. For every image, there are two binary masks: one with the entire breast tissue (to remove the background) and another with the pectoral muscle. For the images with masses, there is also a binary groundtruth image.\par
