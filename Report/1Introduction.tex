\section{Introduction}
\label{sec:intro}
\subsection{Breast Cancer}
Breast cancer is the most commonly diagnosed cancer among women, with approximately 182,000 women diagnosed with the condition annually in the United States \cite{Jemal2009}. Every year, around 40,000 women die of breast cancer, making it the second-leading cause of cancer death among American women after lung cancer.\par
Clinical symptoms of breast cancer include change in breast size or shape, skin changes, nipple abnormalities, single-duct discharge, and axillary lumps \cite{Chalasani2018}. However, those symptoms usually appear late in the course of disease, reflecting advanced staging and less chance of cure. To be able to detect cancer in earlier stages, mammograms must be used.\par

\subsection{Mammograms}
Mammography is a radiographic technique for imaging of the breast. It is used for screening and diagnostic purposes. In both cases, the technique is similar. Four images are acquired, two for each breast, one being medio-lateral oblique view and another craniocaudal view. If necessary, more specific views can be obtained, according to the orientation given by the radiologist. Mammograms have an overall sensibility around 88-93\% and specificity between 85-94\% \cite{Saude2007}. Therefore, they form an excellent method for early diagnosis of breast cancer. \par
An exam report is based on the BI-RADS system \cite{Sickles2013}. BI-RADS was designed to standardize breast imaging reporting and to reduce confusion in breast imaging interpretations. It is also an important tool that allows outcome monitoring and quality assessment. The classification by the BI-RADS varies between categories from 0 to 6. Category 0 indicates the necessity for further imaging investigation, and 6 indicates a mammogram already known to be positive for cancer. The categories between 1 and 5 indicate the risk for cancer, category 1 being very low and 5 very high.\par
The definition of the category relies on the imaging findings. The most common features evaluated by the radiologist include masses, calcifications, architectural distortions, asymmetries, lymphadenopathy. Those findings might be very easy or very hard to be detected, according to their characteristics and the breast composition. Therefore, it is of great clinical value to develop systems that can automatically detect suspicious areas, drawing the radiologist's attention to possible abnormalities. In the presented project, a computer aided diagnosis (CAD) system was developed to detect masses in mammograms.\par